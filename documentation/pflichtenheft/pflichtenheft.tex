
%Schriftgroesse, Layout, Papierformat, Art des Dokumentes
%Hinweis: statt oneside einfach twoside für doppelseitigen druck!
\documentclass[12pt,oneside,a4paper,bibtotoc,liststotoc,pointlessnumbers]{scrartcl}



\usepackage[utf8]{inputenc}
\usepackage[german]{babel}

%etwas schoenere serifen-schrift (mit schnoerkel)
%
%\usepackage[T1]{fontenc}
%\usepackage{palatino}
%\usepackage{microtype}

%bei vielen problemen mit trennung von woertern... macht, dass wörter nicht andauernd getrennt werden
\sloppy

%standard typewriter-font (fuer url oder code) 
\renewcommand\ttdefault{cmtt}

%Wer fuer Ueberschriften eine Schrift ohne Schnoerkel will,
%bitte folgende Zeile aktivieren:
\renewcommand\sfdefault{cmss}

%Ansonsten aktivieren wir die normale Serifen-Schrift fuer die Ueberschriften
%\renewcommand\sfdefault{cmr}

%Ansonsten aktivieren wir die Serifen-Schrift palatino fuerr die Ueberschriften
%oben muss \usepackage{palotino} aktiviert sein!
%\renewcommand\sfdefault{ppl}


%Einstellungen der Seitenraender
\usepackage[inner=3.0cm,outer=3.0cm,top=3cm,bottom=2.5cm]{geometry}

%fürs Binden ...
%\usepackage[inner=3.0cm,outer=3.0cm,top=3cm,bottom=2.5cm]{geometry}

%neue Rechtschreibung
%\usepackage{ngerman}

%Umlaute ermoeglichen, je nach Fileformat UTF8 oder ANSI
%\usepackage[utf8]{inputenc}
%\usepackage[ansinew]{inputenc}

\usepackage[utf8]{inputenc}
\usepackage[german]{babel}

%\usepackage{amsmath}
%\usepackage{txfonts}

%Kopf- und Fusszeile
\usepackage{fancyhdr}
\pagestyle{fancy}
\fancyhf{}
%Kopfzeile links bzw. innen
\fancyhead[LO,RE]{\nouppercase{\leftmark}}
%Kopfzeile rechts bzw. außen
\fancyhead[RO,LE]{\thepage}
%Linie oben
\renewcommand{\headrulewidth}{0.5pt}

%Linie unten - bei mir auskommentiert wegen Fussnoten zur Literatur
%\renewcommand{\footrulewidth}{0.5pt}

%Fuer URLs in einer monospace typewriter font. Verwendung: \url{www.hallo.de}
\usepackage{url}

%externe PDFs einbinden
%\usepackage{pdfpages}
  
%Paket zur Modifizierung der Fussnote. perpage resettet den Counter pro Seite neu
%\usepackage[
 %  bottom,      % Footnotes appear always on bottom. This is necessary
                % especially when floats are used
   %stable,      % Make footnotes stable in section titles
  % perpage,     % Reset on each page
   %para,       % Place footnotes side by side of in one paragraph.
   %side,       % Place footnotes in the margin
   %ragged,      % Use RaggedRight
   %norule,     % suppress rule above footnotes
  % multiple,    % rearrange multiple footnotes intelligent in the text.
   %symbol,     % use symbols instead of numbers
%]%{footmisc} 

%Macht Fuï¿œnotenhinweiï¿œe im wikipedia-Format: hochgestellte kleine Zahl in eckigen Klammern
%Fuer nur hochgestellte kleine Zahlen: diese beiden Zeilen auskommentieren
%\deffootnotemark{\textsuperscript{[\thefootnotemark]}}
%\deffootnote[1em]{1.5em}{1em}{\textsuperscript{[\thefootnotemark]}}

%Fuer die Biblitothekseinbindung, kann so gelassen werden, da der naechste Befehl eigentlich alles regelt
%\usepackage[%
   %round,   %(default) for round parentheses;
 %  square,   % for square brackets;
   %curly,   % for curly braces;
   %angle,   % for angle brackets;
   %colon,   % (default) to separate multiple citations with colons;
  % comma,   % to use commas as separaters;
   %authoryear,% (default) for author-year citations;
   %numbers,   % for numerical citations;
   %super,   % for superscripted numerical citations, as in Nature;
   %sort,      % orders multiple citations into the sequence in which they appear in the list of references;
   %sort&compress,    % as sort but in addition multiple numerical citations
                   % are compressed if possible (as 3-6, 15);
   %longnamesfirst,  % makes the first citation of any reference the equivalent of
                   % the starred variant (full author list) and subsequent citations
                   %normal (abbreviated list);
   %sectionbib,      % redefines \thebibliography to issue \section* instead of \chapter*;
                   % valid only for classes with a \chapter command;
                   % to be used with the chapterbib package;
   %nonamebreak,     % keeps all the authors names in a citation on one line;
                   %causes overfull hboxes but helps with some hyperref problems.
%]{natbib} 

%Festlegung Art der Zitierung - alphanumerische DIN-Methode: Abkuerzung Autor + Jahr
%\bibliographystyle{alphadin}

%zum Zitieren
%\usepackage{cite}

%Fuer das Glossar - noch nicht Benutzt...
%\usepackage[toc,acronym, section=section, nonumberlist]{glossaries}

%Fuer 1,5-fachen Zeilenabstand
%\RequirePackage{setspace}
%\onehalfspacing

%folgende Woerter werden von latex NICHT mehr getrennt, sehr praktisch
%mehrere woerter durch leerzeichen trennen, fuer mehrsilbige woerter kann 
%man mit - an den "trennpunkten" unterteilen
%\hyphenation{Java Giesecke Devrient Hash-funk-tion Hash-funk-tionen Hash-wert CryptoApi PCGQueueServer PCGCryptoApi CryptoApi PCGRequestQueue PCGDatabaseUtil}


\setlength{\abovecaptionskip}{1.0ex} % Abstand ueber Bildbeschriftung
\setlength{\belowcaptionskip}{0.5ex} % Abstand unter Bildbeschriftung
\setlength{\parindent}{0pt}			% Absatzeinrueckung (keine)
\setlength{\parskip}{6pt}		

%Inhaltsverzeichnis: kleine roemische Zahlen
\newenvironment{inhaltsverzeichnis}{
\pagenumbering{roman}	
%\pagestyle{fancy}
}


\usepackage{listings}		% fuerr Quelltext
\usepackage{caption3}		% Bildbeschriftung
\usepackage{xcolor} % Farbe
\usepackage{graphicx} % -> bilder!

% Fussnotenlinie
%Formatierungsregeln fuer Listings
\lstset{%
language=java, %Setzt die Sprache
basicstyle=\scriptsize,% Setzt den Standardstil
keywordstyle=\color{black}\bfseries, % Setzt den Stil fuer Schluesselwoerter
identifierstyle=, % Identifier bekommen keine gesonderte formatierung
commentstyle=\color{black}, % Stil fuer Kommentare
stringstyle=\ttfamily, % Stil fuer Strings (gekennzeichnet mit"String")
breaklines=true, % Zeilen werden umgebrochen
numbers=left, % Zeilennummern links
numberstyle=\tiny, % Stil fuer die Seitennummern
frame=single, % Rahmen
backgroundcolor=\color{white}, % Hintergrundfarbe
%caption={Java-Code} % Caption
}

\definecolor{darkblue}{rgb}{0,0,.5}
\definecolor{darkred}{rgb}{.8,0,0}

\usepackage[%
		pdftex,
		colorlinks,
 %  bookmarks, 
    bookmarksnumbered=true, 
    bookmarksopen=true, 
    bookmarksopenlevel=1,
%    hyperfootnotes=true,
		linkcolor=darkred, %standard red,
		citecolor=darkred, %standard green
		urlcolor=darkred, %standard cyan
		filecolor=darkred, %  
   % menubordercolor={0 1 1},    
   % urlbordercolor={1 0 0}      
 %   hyperfootnotes=true,
 %   hyperindex=true,
 %   pdfpagelayout=OneColumn, 
 %   plainpages=false, 
 %   pdfpagelabels,
    pdfusetitle,
    pdfstartpage={1},
    pdfstartview={FitV},
  %  pdfauthor={},
  %  pdftitle={Bachelorarbeit}  
]{hyperref}

\renewcommand{\footnoterule}{\vfill\rule{7.3cm}{0.5pt}\vspace{1ex}}	

%Erneuerung des Namens fuer das Listingsverzeichnis von "Listings" -> "Listingsverzeichnis"
\renewcommand{\lstlistlistingname}{Listingsverzeichnis}

\renewcommand\figurename{Abb.}

%\renewcommand\listfigurename{Diagrammverzeichnis}}

%Erneuerung des Namens fuer das Literaturverzeichnis von "Literatur" -> "Literaturverzeichnis"
\renewcommand\refname{Literaturverzeichnis}

\usepackage{epstopdf}
%\usepackage{subfigure}
%Los gehts
\begin{document}


%%Titelseite -> muss durch FH-Vorlage ersetzt oder an diese angepasst werden
%\title{
%\includegraphics[scale=0.16]{hs.pdf} \qquad  \includegraphics[scale=0.35]{pics/gd.eps}  \\ \vspace*{50pt}
%Bachelorarbeit in Informatik \\ \vspace*{30pt} \textbf{Generierung sicherer Produktcodes in einer serviceorientieren Client-Server-Architektur} \\ 
%%\includegraphics[scale=1.0]{pics/giesecke_devrient_ger.png} \\
%}
%\author{Benedikt Lippert}
%\thispagestyle{empty}
%\maketitle
%\thispagestyle{empty}

%\hypertarget{title}{}
%\pdfbookmark[1]{Titelblatt}{title}
\thispagestyle{empty}

\begin{center}
\includegraphics[scale=0.16]{hs.pdf}\\
\vspace*{10pt}
\textsf{\textbf{\large{Hochschule Landshut}}}\\
\textsf{\normalsize{Fakultät Informatik \\ Studiengang Informatik (Master)}}\\
%Falls noch ein Firmenlogo reinsoll, folgenden Block einkommentieren und Bild und Titel tauschen.
%Achtung: vspaces müssen dann angepasst werden danach, sonst rutscht alles unten raus
%\vspace*{40pt}
%\includegraphics[scale=0.16]{pics/hs.pdf}\\
%\vspace*{10pt}
%\textbf{\large{Firma}}\\
%\normalsize{ggf. Abteilung}\\
\vspace*{70pt}
\textsf{\textbf{\Huge{Pflichtenheft zum Studienprojekt}}} \\
\vspace*{20pt}
\textsf{\textbf{\large{Verteilter Dateisynchronisationsdienst \vspace*{20pt} Synchrony}}}\\
\end{center}
\vspace*{180pt}

\textsf{\large{Stand: 31.05.2010}}\\
\vspace*{20pt}
\\ 

Teilnehmer: \hspace*{0.5cm}Simon Bauer, Michael Bien, Markus Grundl, \\
\hspace*{2.75cm}Siegfried Hable, Benedikt Lippert \\

Betreuer: \hspace*{0.95cm}Prof. Dr. Wolfgang Jürgensen\\




\newpage
\thispagestyle{empty}
%\mbox{}
%\newpage


%\hypertarget{abstract}{}
%\pdfbookmark[1]{Kurzzusammenfassung / Abstract}{abstract}
\thispagestyle{empty}
\section*{Kurzzusammenfassung}
Dies ist das Pflichtenheft zum Studienprojekt des Master-Studiengangs SS2010/WS2011 mit dem Titel "Verteilter Dateisynchronisationsdienst".\par
Dieses Pflichtenheft ist nach dem Gliederungsvorschlag von Prof. Dr. Helmut Balzert (Ruhr-Universität Bochum) aus seinem Buch "Lehrbuch der Softwaretechnik"\footnote{Helmut Balzert - Lehrbuch der Softwaretechnik \url{http://amazon.de/o/ASIN/3827417058/}} strukturiert und gibt eine Übersicht der Anforderungen an der in diesem Studienprojekt zu realisierenden Softwarekomponente.
%\section*{Abstract}
%Wenn gewünscht, auch noch eine englische Zusammenfassung...

\newpage
%\thispagestyle{empty}
%\mbox{}
%\newpage

\begin{inhaltsverzeichnis}
\clearpage% oder \cleardoublepage bei twoside
\begingroup
  %\pagestyle{plain}
 % \pdfbookmark[1]{Inhaltsverzeichnis}{toc}
  \tableofcontents
  %\cleardoublepage
\endgroup

\end{inhaltsverzeichnis}

\newpage
%\thispagestyle{empty}
%\mbox{}
%\newpage

\pagenumbering{arabic}
\setcounter{page}{1}

%\newpage

\section{Zielbestimmung}
Synchrony ist eine Software zum Synchronsieren von Verzeichnissen und Dateien.
Die Anwendung ermöglicht Datenbestände zwischen verschiedenen Geräten synchron zu halten.
Synchrony stellt eine benutzerfreundliche Benutzeroberfläche zur Verfügung. Diese dient dazu
zu synchronsierende Ordner anzugeben und den Synchronisationsdienst zu aktivieren. Ist der Dienst auf
den jeweiligen Geräten aktiv, so läuft der Abgleich der Daten im Hintergrund ab. Die Verzeichnisse 
werden dabei in beiden Richtungen abgeglichen.
\subsection{Muss-Kriterien}
Die im Folgenden genannten Punkte stellen unabdingbare Produktleistungen dar:
\begin{itemize}
\item Der Benutzer kann Dateien zwischen zwei Rechnern synchronisieren. (kein Mount!)
\item Der Benutzer kann den Synchronisationsdienst konfigurieren.
\item Bei der Installation wird ein Account vergeben, der zur Identifikation im Netzwerk dient.
\item Der Benutzer kann den Synchronisationsdienst aktiveren bzw. deaktivieren.
\item Der Benutzer kann zu die Verzeichnisse, welche er synchronsieren will selbst bestimmen und verändern.
\item Der Benutzer kann den Status der Synchronisation verfolgen.
\item Der Benutzer kann wählen, ob die Oberfläche deutsch oder englischsprachig angezeigt wird.
\item Nach einmaliger Konfiguration wird die Verbindung automatisch hergestellt und die  
  nötigen Updates, nach Dateiänderungen, durchgeführt.
\item Die Dateisynchronisation erfolgt auf Odnerebene, d.h. Dateien der Subordner werden ebenfalls
  synchronsiert
\end{itemize}
\subsection{Optionale Kriterien}
\begin{itemize}
\item Der Synchronsiationsdienst ist automatisch beim Start des Gerätes aktiv.
\item Über einen Webserver wird ein ’Social Sharing’ - Mechanismus realisiert.
\item Eine Anbindung mobiler Clients soll evaluiert werden. (z.B. Android, IPhone)
\item Eine Anbindung an etablierte Protokolle soll evaluiert werden.
\end{itemize}
\subsection{Zielabgrenzung}
Der Anwendungszweck ist nicht identisch mit P2P (Torrent), d.h. kein Filesharing-Protokoll, das sich besonders für die schnelle Verteilung großer Datenmengen eignet.
\newpage
\section{Produkteinsatz}
\subsection{Anwendungsbereich}
Die Softwarekomponente soll Daten jeglicher Art synchron halten und replizieren können, d.h. sie ist auf keinen konkreten Anwendungsbereich limitiert oder auf einen solchen speziell zugeschnitten.\par
Der Dienst ist jedoch, wie in Abschnitt 2.3 beschrieben wird, als Hintergrunddienst gedacht, um Dateien und Informationen zwischen verschiedenen Speicherlokationen zeitnah synchron zu halten und wird daher nicht als Vertriebskanal von großen Dateien (Stichwort Peer-to-Peer-Filesharing) entworfen. n
\subsection{Zielgruppe}
Prinzipiell ist die für jeden Benutzertyp geeignet, die Dateien zwischen mehreren System synchron halten bzw. den Dienst zur Datensicherung verwenden wollen.\par
Nachfolgend eine beispielhafte und nicht abgeschlossene Liste möglicher Anwender und Anwendungsszenarien der Softwarekomponente:
\begin{itemize}
	\item Software-Entwickler: Dateien wie z.B. zur Entwicklung von Software aus einem Projekt-Workspace werden zeitnah abgeglichen, sodass an einem anderen Arbeitsplatz weiterentwickelt werden kann.
	\item Photographen: Beispielsweise können an einem Standort A geschossene Photos über eine mobile Internetverbindung mit einer Workstation am Standort B sofort betrachtet und bearbeitet werden. 
	\item Lehrer/Professoren: Dokumente wie Übungsaufgaben o. Ä. können zwischen Systemen des Büros und des Heimarbeitsplatzes synchron gehalten werden.
	\item Buchauthoren: Skripte, Notizen usw. können zwischen verschiedenen Geräten an verschiedenen Standorten synchronisiert und repliziert werden. 
\end{itemize}
\subsection{Betriebsbedingungen}
Der Dateisynchronisationsdienst soll i.d.R. im Hintergrund als Benutzerdienst laufen und im Hintergrund aktiv sein. Der Dienst wird daher i.d.R. beim Einloggen des Anwenders gestartet.\par
Der Benutzer hat aber auch Möglichkeiten der Steuerung des Dienstes über eine graphische Schnittstelle (GUI), die später im Dokument genauer spezifiziert wird.\par
Das Ausführen der eigentlichen Funktion des Dienstes bedarf aber keiner Aktion des Benutzers.

\newpage
\section{Produktumgebung}
\subsection{Software}
Synchrony stellt folgende Anforderungen an das System hinsichtlich softwareseitiger Unterstützung:
\begin{itemize}
	\item Java Runtime Environment 7 oder höher
	\item ein mit der Dateisystem-API von Java 7 kompatibles Dateisystem
	\item für die Verwendung der graphischen Schnittstelle (GUI) muss ein kompatibles Betriebssystem/Windows-Manager vorhanden sein
\end{itemize}
\subsection{Hardware}
Die Voraussetzungen hinsichtlich der Hardware sind folgendermaßen:
\begin{itemize}
	\item stabile und leistungsfähige Netzwerk- bzw. Internetverbindung, je nach Einsatzszenario
	\item ein ausreichend leistungsfähiges System um Datenkopiervorgänge im Hintergrund durchführen zu können ohne die normale Nutzung zu beeinträchtigen
\end{itemize}
\newpage
\section{Produktfunktionen}
\subsection{Funktionalitäten}
Text: Ziel der SW aus USER-Sicht\\
-Was tut es?\\
-Infü über Aktivität im System-Icon
\subsection{Konfiguration}
-Konfig-file\\
-Einfache GUI zur konfiguration (Änderungen dann im Konfig-File, damit SW-Neustart überlebt werden + Logeintrag)
\newpage
\section{Produktdaten}
- Hash-Trees in .synchrony\\
- Logfile
\newpage
\section{Produktleistungen}
- minimale Zugriffe auf HDD (Systemlast gering halten)\\
- bei Versionskonflikten: - kein Datenverlust\\-Möglichkeit zur interaktiven Konfliktlösung
\newpage
\section{Benutzeroberfläche}
- Systemtray: startet Konfigfenster - start service - stop service - getStatus();\\
-Systemnotifications (BS-abhängig), aber optional
\newpage
\section{Qualitätsbestimmungen}

\begin{table}[h]							
\begin{center}
 \begin{tabular}{l|c|c|c|c}
  ~ & sehr wichtig & wichtig & weniger wichtig & unwichtig\\
  \hline \hline
  Robustheit~ & \textbf{X}~ &  ~ ~ ~ &  ~ ~ ~ &  ~ ~ ~ \\
  \hline
  Zuverlässigkeit~ & \textbf{X}~ &  ~ ~ ~ &  ~ ~ ~ &  ~ ~ ~ \\
  \hline
  Korrektheit~ & \textbf{X}~ &  ~ ~ ~ &  ~ ~ ~ &  ~ ~ ~ \\
  \hline
  Benutzungsfreundlichkeit~ &  ~ ~ ~ & \textbf{X}~ &  ~ ~ ~ &  ~ ~ ~ \\
  \hline
  Effizienz~ &  ~ ~ ~ & \textbf{X}~ &  ~ ~ ~ &  ~ ~ ~ \\
  \hline
  Portierbarkeit~ &  ~ ~ ~ &  ~ ~ ~ & \textbf{X}~ &  ~ ~ ~ \\
  \hline
  Kompatibilität~ &  ~ ~ ~ &  ~ ~ ~ & \textbf{X}~ &  ~ ~ ~ \\
 \end{tabular}
\end{center}
\caption{Matrix der Qualitätsanforderungen}									% Bildbeschriftung
\label{fig:LogoGD}												% Sprungmarke fuer Verweise
\end{table}


\
\newpage
\section{Testfälle}
Die Testfälle werden jeweils vor der Programmierung einer Funktion überlegt und implementiert. Das Prinzip dahinter ist das sogenannte
Test Driven Development, welches die Tests als Basis der Programmierung festlegt. Diese Vorgehensweise setzt von Anfang an die Qualität
der Software in den Mittelpunkt und gewährleistet somit einen hohen Qualitätsstandard. Dadurch sind die Tests automatisch in den 
Entwicklungsprozess mit einbezogen und einzelne Testfälle werden nicht vergessen. Die Tests werden mit sogenannten JUnit-Tests realisiert.
%TEst driven Development; Junit-Tests -> Destdescription
\newpage
\section{Entwicklungsumgebung}
\subsection{Software}
Für die Entwicklung des Synchronisations-Tools synchrony wird die Entwicklungsumgebung NetBeans (Version 6.8) verwendet.
synchrony wird in der Programmiersprache Java entwickelt - dazu wird auf den Entwicklungsrechnern JDK 7 installiert.
Das Release von JDK 7 ist voraussichtlich im dritten Quartal 2010, wird aber aufgrund der neu hinzugekommenen File-API bereits für
dieses Projekt verwendet. Datei-Operationen sind ein wesentlicher Bestandteil für synchrony und diese sind in Java 7 sehr effizient
implementiert.

Das Projektverwaltungswerkzeug Maven 2 (Version 2.2.1) wird installiert und ist zugleich als Plugin in NetBeans (ab Version 6.7) automatisch
integriert. Maven wird zum Erstellen eines Software-Projektes verwendet und bietet eine Vielzahl an Plugins für z. B. die Verwaltung eines
Projektes oder die Durchführung des Buildzyklus. Außerdem bietet Maven ein sehr komfortables Management der benötigten Bibliotheken, welche
aus öffentlichen und privaten Repositories bezogen werden können, d. h. das manuelle Suchen der Bibliotheken entfällt somit.

GitHub (Version 1.6.3.3) wird als Online-Versionskontrollsystem für die Verwaltung von Quellcode eingesetzt. Damit lässt sich die Zusammenarbeit im
Team besser organisieren. GitHub ist eine OpenSource-Software, welche der GNU General Public License unterliegt.

\subsection{Hardware}
Die Entwicklung erfolgt auf unterschiedlichen Rechnern, allerdings mit dem gleichen Betriebssystem Ubuntu 9.04 - 10.04.
Die Rechner-Hardware spielt keine entscheidende Rolle, es muss lediglich die Java Virtual Machine darauf laufen können. 

%Siehe Bsp für 1. Satz; ansonsten ist nur JDK 7 und Maven zu nennen, erklären
\newpage
\section{Ergänzungen}
Lorem ipsum dolor sit amet, consetetur sadipscing elitr, sed diam nonumy eirmod tempor invidunt ut labore et dolore magna aliquyam erat, sed diam voluptua. At vero eos et accusam et justo duo dolores et ea rebum. Stet clita kasd gubergren, no sea takimata sanctus est Lorem ipsum dolor sit amet. Lorem ipsum dolor sit amet, consetetur sadipscing elitr, sed diam nonumy eirmod tempor invidunt ut labore et dolore magna aliquyam erat, sed diam voluptua. At vero eos et accusam et justo duo dolores et ea rebum. Stet clita kasd gubergren, no sea takimata sanctus est Lorem ipsum dolor sit amet.
\newpage
\section{Glossar}
Lorem ipsum dolor sit amet, consetetur sadipscing elitr, sed diam nonumy eirmod tempor invidunt ut labore et dolore magna aliquyam erat, sed diam voluptua. At vero eos et accusam et justo duo dolores et ea rebum. Stet clita kasd gubergren, no sea takimata sanctus est Lorem ipsum dolor sit amet. Lorem ipsum dolor sit amet, consetetur sadipscing elitr, sed diam nonumy eirmod tempor invidunt ut labore et dolore magna aliquyam erat, sed diam voluptua. At vero eos et accusam et justo duo dolores et ea rebum. Stet clita kasd gubergren, no sea takimata sanctus est Lorem ipsum dolor sit amet.

\newpage
%\thispagestyle{empty}
%\mbox{}
%\newpage

%\newpage
\listoffigures

\newpage
%\thispagestyle{empty}
%\mbox{}
%\newpage

%\newpage
\listoftables

%\newpage
%\thispagestyle{empty}
%\mbox{}
%\newpage

%\newpage
%\lstlistoflistings

\newpage
%\thispagestyle{empty}
%\mbox{}
%\newpage

%\setcounter{secnumdepth}{1} 
%Anhang!
\begin{appendix}
\pagenumbering{Roman}	

%\setcounter{figure}{0}


\section{Anhang: der erste Anhang}
AES-Verschlüsselung mit der Sun JCE
Einkommentieren !
%\lstinputlisting[captionpos=b, caption=JCE AES Example]{listings/appendix/1_jcaAesExample.java}
%
%Beispiel HMAC-MD5 aus der Sun JCE
%\lstinputlisting*[captionpos=b, caption=JCE HMAC-MD5 Example]{listings/0_hmacmd5_example.java}

%Listings können, anders als Grafiken und Tabellen nicht
%per * aus dem Listingsverzeichnis herausgelassen werden!
%Falls ein Listing (z. B. im Anhang) nicht in das Verzeichnis
%soll muss getrickst werden: 
%	1. Auskommentieren der Listings im Anhang -> pro Listing ein \newpage,
%	dass die Seitenzahl des Anhangs gleich bleibt
% 2. Setzen, sodass nur noch Listings des eigentlichen Dokuments im VZ
%	WICHTIG: PDF jetzt wegspeichern!
% 3. Änderungen von 1. rückgängig machen, alles setzen, sodass Anhang wieder passt
% Das eigentliche, resultieren Dokument setzt sich jetzt aus dem PDF von 2.
% (inkl. Listingsverzeichnis) + Anhang von 3. zusammen
% Besserer Weg?!?!?

\newpage
%\thispagestyle{empty}
%\mbox{}
%\newpage


\section{Anhang: der zweite Anhang}
\begin{figure}[h]					
\centering
Einkommentieren!
%\includegraphics[scale=0.75]{pics/7_jmx_console.png}		% Grafikeinbindung mit Dateiname und Groessenangabe			% ([width=], [height=], [scale=1.00]...)
\caption{Screenshot der JMX-Weboberfläche}		% Bildbeschriftung
\label{fig:Fabrikmethode}							% Sprungmarke fuer Verweise
\end{figure} 	\vspace{-12pt}

\newpage
%\thispagestyle{empty}
%\mbox{}
%\newpage



\newpage
\section{Anhang: der dritte Anhang}
%Fussnoten-Tests:
%\footnote{\citep[vgl. ][S. 1]{Schmeh2009}}
%\footnote{\citep[vgl. ][S. 23]{Schmeh2009}}
%\footnote{\citep[vgl. ][S. 1-100]{Schneier1996}}
%\footnote{\citep[vgl. ][S. 1]{Monson-Haefel2009}}
%\citep[vgl. ][S. 1]{Langner2006}
%\citep[vgl. ][S. 1]{Lipp2000}
%\citep[vgl. ][S. 1]{Backschat2007}
%\footnote{\citep[vgl. ][]{Schneier1996}}

%nocite bewirkt,d ass das ganze bib geschrieben wird, auch wenn noch keine cites (referenz) im text auf jedes element existieren
\nocite{*}

\newpage
Folgendes Bild ist im Abbildungsverzeichnis:
\begin{figure}[h]							
\centering 
Einkommentieren!
%\includegraphics[scale=1.0]{pics/giesecke_devrient_ger.png}		% Grafikeinbindung mit Dateiname und Groessenangabe												% ([width=], [height=], [scale=1.00]...)
\caption{Logo von Giesecke \& Devrient}									% Bildbeschriftung
\label{fig:LogoGD}												% Sprungmarke fuer Verweise
\end{figure}

Folgendes Bild ist NICHT im Abbildungsverzeichnis: \\ %einfach  den figure-mantel weglassen
\begin{center}
Einkommentieren!
%\includegraphics*[scale=1.0]{pics/giesecke_devrient_ger.png}	
\end{center}

Tabellenobjekt: hier ist ein Bild in einer Tabelle mit einer Spalte und einer Zelle:
So können z.B. Tabellen, die als Bilddatei vorliegen als Tabelle betitelt und auch ins Tabellen-VZ aufgenommen werden.
%\begin{table}[h]							
%\centering 
%Einkommentieren!
%\includegraphics[scale=1.0]{pics/giesecke_devrient_ger.png}		% Grafikeinbindung mit Dateiname und Groessenangabe													% ([width=], [height=], [scale=1.00]...)
%
%\caption{Logo von Giesecke \& Devrient als Tabelle deklariert}									% Bildbeschriftung
%\label{fig:LogoGD}												% Sprungmarke fuer Verweise
%\end{table}




\end{appendix}
%\end{anhang}

%\newpage
%\thispagestyle{empty}
%\mbox{}

\end{document}
